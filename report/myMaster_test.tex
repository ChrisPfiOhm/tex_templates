% #################################################
\newcommand{\titel}{Titel dieser Arbeit}
\newcommand{\untertitel}{Untertitel dieser Arbeit}
\newcommand{\art}{Bachelor Masterarbeit}
\newcommand{\fachgebiet}{Elektro- und Informationstechnik}
\newcommand{\autortitle}{B.\,Eng. }
\newcommand{\autor}{Peter M\"{u}ller}
\newcommand{\studienbereich}{Elektro- und Informationstechnik}
\newcommand{\matrikelnr}{123\,456\,78}
\newcommand{\jahr}{\the\year}
\newcommand{\ort}{Nürnberg}
\newcommand{\hochschule}{Technische Hochschule N\"{u}rnberg \\Georg Simon Ohm}
\newcommand{\fachbereich}{Elektro- und Informationstechnik}
\newcommand{\betreuer}{Prof. Dr. Stefan Test}
\newcommand{\schlagwortA}{SchlagwortA}
\newcommand{\schlagwortB}{SchlagwortB}
\newcommand{\schlagwortC}{SchlagwortC}
\newcommand{\schlagwortD}{SchlagwortD}     

\newcommand{\zitat}{	"Jede hinreichend fortgeschrittene Technologie \\ ist von Magie nicht mehr zu unterscheiden."\\[2ex]
	\textit{Sir Arthur Charles Clarke  (1917-2008)}, \\
	 Britischer Science-Fiction Autor, Erfinder, and Futurist}
\newcommand{\vorwort}{Ich danke allen die mir geholfen haben. } %
\newcommand{\zusammenfassungDE}{Lorem ipsum dolor sit amet, consetetur sadipscing elitr, sed diam nonumy eirmod tempor invidunt ut labore et dolore magna aliquyam erat, sed diam voluptua.} %
\newcommand{\zusammenfassungEN}{Lorem ipsum dolor sit amet, consetetur sadipscing elitr, sed diam nonumy eirmod tempor invidunt ut labore et dolore magna aliquyam erat, sed diam voluptua.} %

\def\textLanguage{ger}  % ger oder eng
\def\print{true}	   % true oder false

% ################################################
\documentclass[
    11pt,										
    DIV10,
    ngerman, 							
    a4paper, 												    		
    %twoside,
    oneside,   								
    %openright,								
    onecolumn, 
    titlepage, 							
    parskip=half, 					
    headings=normal, 
    listof=totoc, 						
    bibliography=totoc, 			
    index=totoc, 	
    captions=tableheading, 	
    final 										
]{my_Master}

\graphicspath{/home/chris/Dropbox/Forschungsmaster/workspace/pic/png_jpg} 

% #################################################
% Document
% #################################################
\begin{document}

% auch subsubsection nummerieren
\setcounter{secnumdepth}{3}
\setcounter{tocdepth}{3}

% Deckblatt und Abstract ohne Seitenzahl
\ifoot{}
\ofoot{}
\input{Verzeichnisse/Deckblatt}
\input{Verzeichnisse/Abstract}

\ofoot{\pagemark}

\pagenumbering{Roman}	
\tableofcontents 		

% Abkürzungsverzeichnis --------------------------------------------------------
\input{Verzeichnisse/Glossar}
%\input{Verzeichnisse/Abkuerzungen}
\label{sec:Glossar}

\ifx \textLanguage\eng
	\addcontentsline{toc}{chapter}{List of Abbreviations}
\else
	\addcontentsline{toc}{chapter}{Abkürzungsverzeichnis}	
\fi
\listoffigures 	
\listoftables 	

% arabische Seitenzahlen im Hauptteil --------------------------------------\ifx \textLanguage\eng----
\clearpage
\pagenumbering{arabic}
% \spacing{1.3}

% #################################################
% Inhalt 
% #################################################
\chapter{Einleitung}
\label{cha:Einleitung}

\section{Motivation}

\section{Ziel der Arbeit und Aufgabenstellung}
Lorem ipsum dolor sit amet, consetetur sadipscing elitr, sed diam nonumy eirmod tempor invidunt ut labore et dolore magna aliquyam erat, sed diam voluptua. At vero eos et accusam et justo duo dolores et ea rebum. Stet clita kasd gubergren, no sea takimata sanctus est Lorem ipsum dolor sit amet. Lorem ipsum dolor sit amet, consetetur sadipscing elitr, sed diam nonumy eirmod tempor invidunt ut labore et dolore magna aliquyam erat, sed diam voluptua. At vero eos et accusam et justo duo dolores et ea rebum. Stet clita kasd gubergren, no sea takimata sanctus est Lorem ipsum dolor sit amet. Lorem ipsum dolor sit amet, consetetur sadipscing elitr, sed diam nonumy eirmod tempor invidunt ut labore et dolore magna aliquyam erat, sed diam voluptua. At vero eos et accusam et justo duo dolores et ea rebum. Stet clita kasd gubergren, no sea takimata sanctus est Lorem ipsum dolor sit amet.   

%########################################################################################
\section{Typographische Konventionen}
In diesem Abschnitt findet der Leser die \idx{typographische Konvention} für regelmäßig angewendete Formatierungen im Fließtext. Eigen- und Produktnamen sind klein geschrieben, um eine bessere Lesbarkeit zu erzielen.
\begin{table}[htbp]
\begin{center}
\begin{tabular}[ht]{ll}
  \toprule 
  \textbf{Formatierung} & \textbf{Bedeutung}\\
  \midrule
  \emph{Beispiel} &  Hervorgehobener Fachbegriff/Ausdruck \\
%  \color{black}Beispiel & \color{black} Verweis auf Seiten, Abbildung oder Tabelle\\
  \color{black}(Beispiel) & \color{black} Verweis auf Formel\\
  \color{black}(Beispiel, 2012) & \color{black} Verweis auf Literatur mit Ver\-öffent\-lichungs\-datum\\
  \color{black}\Code{Beispiel}		& \color{black} Einstellparameter oder Codesegment\\
  $\vec{v}$ & Vektor \\
  $\mat{M}$ & Matrix \\
  $\vecF{F}$ & Kraft \\
  \bottomrule 
\end{tabular}
\caption{Typographische Konvention.}
\end{center}
\end{table}

Bei Formeln gelten die SI-Einheiten, falls keine andere Deklaration vorgenommen wurde.

\section{Aufbau dieser Arbeit}
Dieser Projektbericht untergliedert sich wie folgt: 
\begin{itemize}
\item \emph{Kapitel 2} Lorem ipsum dolor sit amet, consetetur sadipscing elitr, sed diam nonumy eirmod tempor invidunt ut labore et dolore magna aliquyam erat, sed diam voluptua.
\item \emph{Kapitel 3} Lorem ipsum dolor sit amet, consetetur sadipscing elitr, sed diam nonumy eirmod tempor invidunt ut labore et dolore magna aliquyam erat, sed diam voluptua.
\item \emph{Kapitel 4} Lorem ipsum dolor sit amet, consetetur sadipscing elitr, sed diam nonumy eirmod tempor invidunt ut labore et dolore magna aliquyam erat, sed diam voluptua.
\item \emph{Kapitel 5} Lorem ipsum dolor sit amet, consetetur sadipscing elitr, sed diam nonumy eirmod tempor invidunt ut labore et dolore magna aliquyam erat, sed diam voluptua.
\item \emph{Kapitel 6} Lorem ipsum dolor sit amet, consetetur sadipscing elitr, sed diam nonumy eirmod tempor invidunt ut labore et dolore magna aliquyam erat, sed diam voluptua.
\end{itemize}

% #################################################
% Literaturverzeichnis 
% #################################################
\nocite{*}
\bibliographystyle{plain} 
\inputencoding{latin2}
\bibliography{Bibliographie}
\inputencoding{utf8}

% #################################################
% Anhang 
% #################################################
\begin{appendix}
    \clearpage    
    \pagenumbering{roman}
    \setdefaultleftmargin{1em}{}{}{}{}{}     
    %\input{Anhang}
\end{appendix}

\end{document}
